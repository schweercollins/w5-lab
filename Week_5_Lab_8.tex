\documentclass[man]{apa6}
\usepackage{lmodern}
\usepackage{amssymb,amsmath}
\usepackage{ifxetex,ifluatex}
\usepackage{fixltx2e} % provides \textsubscript
\ifnum 0\ifxetex 1\fi\ifluatex 1\fi=0 % if pdftex
  \usepackage[T1]{fontenc}
  \usepackage[utf8]{inputenc}
\else % if luatex or xelatex
  \ifxetex
    \usepackage{mathspec}
  \else
    \usepackage{fontspec}
  \fi
  \defaultfontfeatures{Ligatures=TeX,Scale=MatchLowercase}
\fi
% use upquote if available, for straight quotes in verbatim environments
\IfFileExists{upquote.sty}{\usepackage{upquote}}{}
% use microtype if available
\IfFileExists{microtype.sty}{%
\usepackage{microtype}
\UseMicrotypeSet[protrusion]{basicmath} % disable protrusion for tt fonts
}{}
\usepackage{hyperref}
\hypersetup{unicode=true,
            pdftitle={Week 5 Lab 8},
            pdfauthor={Shaina Trevino, Maria Schweer-Collins, Alejandra Garcia Isaza, \& Jonathan Pedroza},
            pdfkeywords={Trains, Planes, Automobiles},
            pdfborder={0 0 0},
            breaklinks=true}
\urlstyle{same}  % don't use monospace font for urls
\usepackage{graphicx,grffile}
\makeatletter
\def\maxwidth{\ifdim\Gin@nat@width>\linewidth\linewidth\else\Gin@nat@width\fi}
\def\maxheight{\ifdim\Gin@nat@height>\textheight\textheight\else\Gin@nat@height\fi}
\makeatother
% Scale images if necessary, so that they will not overflow the page
% margins by default, and it is still possible to overwrite the defaults
% using explicit options in \includegraphics[width, height, ...]{}
\setkeys{Gin}{width=\maxwidth,height=\maxheight,keepaspectratio}
\IfFileExists{parskip.sty}{%
\usepackage{parskip}
}{% else
\setlength{\parindent}{0pt}
\setlength{\parskip}{6pt plus 2pt minus 1pt}
}
\setlength{\emergencystretch}{3em}  % prevent overfull lines
\providecommand{\tightlist}{%
  \setlength{\itemsep}{0pt}\setlength{\parskip}{0pt}}
\setcounter{secnumdepth}{0}
% Redefines (sub)paragraphs to behave more like sections
\ifx\paragraph\undefined\else
\let\oldparagraph\paragraph
\renewcommand{\paragraph}[1]{\oldparagraph{#1}\mbox{}}
\fi
\ifx\subparagraph\undefined\else
\let\oldsubparagraph\subparagraph
\renewcommand{\subparagraph}[1]{\oldsubparagraph{#1}\mbox{}}
\fi

%%% Use protect on footnotes to avoid problems with footnotes in titles
\let\rmarkdownfootnote\footnote%
\def\footnote{\protect\rmarkdownfootnote}


  \title{Week 5 Lab 8}
    \author{Shaina Trevino\textsuperscript{1}, Maria
Schweer-Collins\textsuperscript{1}, Alejandra Garcia
Isaza\textsuperscript{1}, \& Jonathan Pedroza\textsuperscript{1}}
    \date{}
  
\shorttitle{Wk5Lb8}
\affiliation{
\vspace{0.5cm}
\textsuperscript{1} University of Oregon}
\keywords{Trains, Planes, Automobiles}
\usepackage{csquotes}
\usepackage{upgreek}
\captionsetup{font=singlespacing,justification=justified}

\usepackage{longtable}
\usepackage{lscape}
\usepackage{multirow}
\usepackage{tabularx}
\usepackage[flushleft]{threeparttable}
\usepackage{threeparttablex}

\newenvironment{lltable}{\begin{landscape}\begin{center}\begin{ThreePartTable}}{\end{ThreePartTable}\end{center}\end{landscape}}

\makeatletter
\newcommand\LastLTentrywidth{1em}
\newlength\longtablewidth
\setlength{\longtablewidth}{1in}
\newcommand{\getlongtablewidth}{\begingroup \ifcsname LT@\roman{LT@tables}\endcsname \global\longtablewidth=0pt \renewcommand{\LT@entry}[2]{\global\advance\longtablewidth by ##2\relax\gdef\LastLTentrywidth{##2}}\@nameuse{LT@\roman{LT@tables}} \fi \endgroup}


\DeclareDelayedFloatFlavor{ThreePartTable}{table}
\DeclareDelayedFloatFlavor{lltable}{table}
\DeclareDelayedFloatFlavor*{longtable}{table}
\makeatletter
\renewcommand{\efloat@iwrite}[1]{\immediate\expandafter\protected@write\csname efloat@post#1\endcsname{}}
\makeatother

\authornote{

Correspondence concerning this article should be addressed to Shaina
Trevino, Postal address. E-mail:
\href{mailto:my@email.com}{\nolinkurl{my@email.com}}}

\abstract{
If You Have Two Loaves of Bread, Sell One and Buy a Lily - A Chinese
Proverb


}

\begin{document}
\maketitle

\section{Methods}\label{methods}

We report how we determined our sample size, all data exclusions (if
any), all manipulations, and all measures in the study.

\subsection{Participants}\label{participants}

\subsection{Material}\label{material}

\subsection{Procedure}\label{procedure}

\subsection{Data analysis}\label{data-analysis}

\section{Results}\label{results}

\section{Discussion}\label{discussion}

Child maltreatment is characterized by its multifinality, this means
that individual differences play a role in how child maltreatment is
experienced and how it affects later development. In part, this can be
explained by timing, dose, chronicity and type of maltreatment (Gunnar,
Fisher, \& others, 2006). For instance, Cowell, Cicchetti, Rogosch \&
Toth (2015) found that on measures of inhibitory control and working
memory, maltreated children showed poorer performance than their
non-maltreated counterparts. Furthermore, Cowell et al. (2015) found
that within the maltreated group, children that had experienced
maltreatment during infancy had worse performance in comparison with
children that experienced maltreatment later in childhood.

\newpage

\section{References}\label{references}

\begingroup
\setlength{\parindent}{-0.5in} \setlength{\leftskip}{0.5in}

\hypertarget{refs}{}
\hypertarget{ref-cowell2015childhood}{}
Cowell, R. A., Cicchetti, D., Rogosch, F. A., \& Toth, S. L. (2015).
Childhood maltreatment and its effect on neurocognitive functioning:
Timing and chronicity matter. \emph{Development and Psychopathology},
\emph{27}(2), 521--533.

\hypertarget{ref-gunnar2006bringing}{}
Gunnar, M. R., Fisher, P. A., \& others. (2006). Bringing basic research
on early experience and stress neurobiology to bear on preventive
interventions for neglected and maltreated children. \emph{Development
and Psychopathology}, \emph{18}(3), 651--677.

\endgroup


\end{document}
